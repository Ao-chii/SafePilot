\documentclass[a4paper,12pt]{article}
\usepackage{geometry}
\usepackage{ctex} % 使用 ctex 宏包,与需求文档统一,自动处理中文字体
\geometry{left=2.5cm,right=2.5cm,top=2.5cm,bottom=2.5cm}
\usepackage{amsmath}
\usepackage{amsfonts}
\usepackage{graphicx}
\usepackage{array}
\usepackage{booktabs}
\usepackage{enumitem}
\usepackage{xcolor}
\usepackage{listings}
\usepackage{hyperref}
\usepackage{titling}
\usepackage{longtable}
\usepackage{float}
\usepackage{tabularx} 
\usepackage{ltxtable}

% 为 listings 环境设置样式,ctex 环境下无需特殊处理中文
\lstset{
  basicstyle=\ttfamily\small,
  breaklines=true,
  frame=single,
  tabsize=2,
  showspaces=false,
  showstringspaces=false,
  commentstyle=\color{gray},
  language=bash, % 将 # 识别为注释
  literate=
    {├}{{\smash{\raisebox{-.5ex}{\rule{1pt}{2.5ex}}}\raisebox{-.5ex}{\rule{0.5em}{1pt}}}}1
    {─}{{\rule{0.5em}{1pt}}}1
    {│}{{\smash{\raisebox{-.5ex}{\rule{1pt}{2.5ex}}}}}1
    {└}{{\smash{\raisebox{0.5ex}{\rule{1pt}{1.5ex}}\llap{\rule{0.5em}{1pt}}}}}1
}

% Title and document setup
\title{驾驶员危险行为检测系统架构设计文档}
\author{}
\date{}

\begin{document}
\maketitle
\tableofcontents
\newpage

\section{架构样式展示}
根据系统需求,本项目采用复合架构风格,以三层架构为核心,确保逻辑清晰、职责分离,同时融入客户端-服务器通信模式,支持实时性和分布式部署。

\subsection{宏观风格:三层架构}
系统在逻辑上被清晰地划分为三个层次,确保了各层职责的独立性,便于开发和维护。
\begin{itemize}
  \item \textbf{表现层}:负责用户交互和数据展示。这包括运行在 PC 端的 \textbf{Vue.js 前端 Web 界面},以及运行在客户端 Python 程序中,通过 OpenCV 实时显示的视频窗口和报警提示。
  \item \textbf{逻辑层}:负责处理业务逻辑和数据。这主要是指 \textbf{Flask 后端服务器},它提供 RESTful API 接口,处理数据存储、查询、统计分析等核心业务。
  \item \textbf{数据层}:负责数据的持久化存储和管理。由 \textbf{PostgreSQL 数据库} 构成。
\end{itemize}

\subsection{通信风格:客户端-服务器架构}
\begin{itemize}
  \item \textbf{智能胖客户端(监测端)}:基于Python的实时监测应用程序,具备完整的AI推理能力。
    \begin{itemize}
      \item 本地执行YOLO v11目标检测和MediaPipe面部关键点分析
      \item 实现多策略行为分析算法(EAR、MAR、PERCLOS)
      \item 提供毫秒级实时报警响应
      \item 支持离线运行和网络中断容错
      \item 采用观察者模式实现组件解耦
    \end{itemize}
  \item \textbf{轻量瘦客户端(管理端)}:基于Vue.js的Web管理界面。
    \begin{itemize}
      \item 纯前端渲染,所有业务逻辑由服务器处理
      \item 响应式设计,支持多设备访问
      \item 实时数据展示和交互式图表
    \end{itemize}
  \item \textbf{中心服务器}:Flask RESTful API服务器,采用微服务化设计思路。
    \begin{itemize}
      \item 事件数据接收与持久化
      \item 用户认证与权限管理
      \item 统计分析与报表生成
      \item 系统监控与健康检查
    \end{itemize}
\end{itemize}

\begin{figure}[h]
  \centering
  \includegraphics[width=0.6\textwidth]{figures/three_tier_architecture.png}
  \caption{三层架构与客户端-服务器架构图}
  \label{fig:three_tier}
\end{figure}

\section{参考模型说明}
为全面描述系统架构,我们选用 \textbf{4+1 视图模型} 作为理论框架。该模型通过多视角描绘系统,确保利益相关者(如开发者关注模块化、管理员关注可用性)获得针对性信息,避免单一视图的偏差。
\begin{enumerate}
  \item \textbf{逻辑视图(Logical View)}:描述系统的功能结构,主要关注系统为用户提供的服务和模块间的责任分工。
  \item \textbf{过程视图(Process View)}:描述系统的动态行为,即系统运行时各个进程、线程间的交互、并发和数据流。
  \item \textbf{开发视图(Development View)}:描述系统的模块划分和组织,包括代码结构、依赖关系和开发规约。
  \item \textbf{物理视图(Physical View)}:描述系统如何部署到硬件节点上,以及各节点间的物理连接和网络拓扑。
  \item \textbf{场景视图(Scenario View)}:作为核心,用于连接和验证其他四个视图,通过具体的用例或场景来展示架构的合理性。
\end{enumerate}

\section{参考架构分析}
应用 4+1 视图模型,从多视角剖析系统架构,确保覆盖功能、动态、开发与部署维度。

\subsection{逻辑视图}
此视图展示了系统的主要功能模块及其关系,采用分层设计和设计模式组合。

\subsubsection{核心类与接口设计}
\begin{itemize}
  \item \textbf{检测层}:
    \begin{itemize}
      \item \texttt{YOLODetector}:采用单例模式,封装YOLOv11模型加载与推理
      \item \texttt{FaceDetector}:基于MediaPipe的面部关键点检测器
      \item \texttt{DetectorFactory}:工厂模式管理检测器实例创建
    \end{itemize}
  \item \textbf{分析层}:
    \begin{itemize}
      \item \texttt{BehaviorAnalyzer}:主分析器,採用观察者模式通知多个监听器
      \item \texttt{BehaviorDetectionStrategy}:策略模式接口,定义行为检测算法
      \item \texttt{EyeStateStrategy}、\texttt{MouthStateStrategy}、\texttt{DistractionStrategy}:具体策略实现
    \end{itemize}
  \item \textbf{响应层}:
    \begin{itemize}
      \item \texttt{AlarmManager}:报警管理器,实现\texttt{BehaviorObserver}接口
      \item \texttt{DataUploader}:数据上传组件,支持批量上传和重试机制
      \item \texttt{ConfigManager}:配置管理器,采用单例模式
    \end{itemize}
\end{itemize}

\subsubsection{设计模式应用关系}
\begin{itemize}
  \item \textbf{观察者模式}:\texttt{BehaviorAnalyzer} $\to$ \texttt{AlarmManager}, \texttt{DataUploader}
  \item \textbf{策略模式}:\texttt{BehaviorAnalyzer} $\to$ \texttt{BehaviorDetectionStrategy} $\to$ 具体策略实现
  \item \textbf{单例模式}:\texttt{YOLODetector}、\texttt{FaceDetector}、\texttt{ConfigManager}
  \item \textbf{工厂模式}:\texttt{DetectorFactory} 管理检测器生命周期
\end{itemize}

\begin{figure}[h]
  \centering
  \includegraphics[width=0.5\textwidth]{figures/logical_view.png}
  \caption{逻辑视图类图}
  \label{fig:logical_view}
\end{figure}

\subsection{过程视图}
此视图通过时序图和活动图展示“检测到危险行为并响应”的动态过程。

\subsubsection{主流程时序图}
以下时序图展示了从视频采集到报警响应的完整流程:

\begin{itemize}
  \item \textbf{初始化阶段}:
    \begin{enumerate}
      \item \texttt{SafePilotApp} 启动并初始化所有组件
      \item 加载YOLO和MediaPipe模型(单例模式保证只加载一次)
      \item 注册观察者:\texttt{AlarmManager}和\texttt{DataUploader}
      \item 初始化视频流\texttt{VideoStreamer}
    \end{enumerate}
  \item \textbf{实时检测循环}(在独立线程中运行):
    \begin{enumerate}
      \item 视频帧采集:\texttt{VideoStreamer.read()}
      \item 图像预处理:\texttt{VideoProcessor.preprocess\_frame()}
      \item 并行AI推理:
        \begin{itemize}
          \item \texttt{YOLODetector.detect()} 检测物体(手机、香烟等)
          \item \texttt{FaceDetector.detect\_face()} 检测面部关键点
        \end{itemize}
      \item 结果合并与格式化
      \item \texttt{BehaviorAnalyzer.analyze()} 执行多策略行为分析
    \end{enumerate}
  \item \textbf{危险行为响应}(观察者模式触发):
    \begin{enumerate}
      \item \texttt{BehaviorAnalyzer} 检测到危险行为后通知所有观察者
      \item \texttt{AlarmManager.on\_behavior\_detected()}:
        \begin{itemize}
          \item 触发声音报警(pygame播放音频)
          \item 显示视觉报警(OpenCV绘制警告框)
          \item 保存报警时刻的视频帧
        \end{itemize}
      \item \texttt{DataUploader.on\_behavior\_detected()}:
        \begin{itemize}
          \item 将事件数据加入本地缓存队列
          \item 异步批量上传至服务器
          \item 实现指数退避重试机制
        \end{itemize}
    \end{enumerate}
\end{itemize}

\subsubsection{并发与线程模型}
\begin{itemize}
  \item \textbf{主线程}:用户界面与OpenCV显示
  \item \textbf{检测线程}:实时AI推理和行为分析
  \item \textbf{上传线程}:异步数据上传,避免阻塞主流程
  \item \textbf{线程同步}:使用\texttt{threading.Lock}保证共享数据的线程安全
\end{itemize}

\begin{figure}[h]
  \centering
  \includegraphics[width=0.8\textwidth]{figures/process_view.png}
  \caption{过程视图时序图}
  \label{fig:process_view}
\end{figure}

\subsection{开发视图}
此视图展示了项目的代码组织结构(见于图4)和模块依赖关系。

\subsubsection{模块依赖关系}
\textbf{客户端模块依赖}:
\begin{itemize}
  \item \texttt{application.py} $\leftarrow$ 聚合所有组件,作为门面模式(Facade Pattern)
  \item \texttt{detector.py} $\leftarrow$ \texttt{analyzer.py} (策略模式依赖)
  \item \texttt{alarm\_manager.py} $\leftarrow$ \texttt{analyzer.py} (观察者模式依赖)
  \item \texttt{data\_uploader.py} $\leftarrow$ \texttt{analyzer.py} (观察者模式依赖)
  \item \texttt{config.py} $\rightarrow$ 全局配置依赖,单例模式
\end{itemize}

\textbf{服务器端模块依赖}:
\begin{itemize}
  \item \texttt{api.py} $\rightarrow$ \texttt{models.py} (数据访问层依赖)
  \item \texttt{models.py} $\rightarrow$ SQLAlchemy ORM $\rightarrow$ PostgreSQL
  \item Flask蓝图(Blueprint)模式实现API组织
\end{itemize}

\subsubsection{构建与部署流程}
\begin{itemize}
  \item \textbf{客户端打包}:使用PyInstaller生成可执行文件,内嵌模型权重
  \item \textbf{服务器部署}:基于Gunicorn + Nginx的生产环境部署
  \item \textbf{前端构建}:Vue CLI构建为静态文件,由Nginx提供服务
  \item \textbf{CI/CD}:预留GitHub Actions或Jenkins集成
\end{itemize}

\begin{figure}[h]
  \centering
  \includegraphics[width=0.8\textwidth]{figures/development_view.png}
  \caption{代码结构}
  \label{fig:development_view}
\end{figure}

\subsection{物理视图}
此视图描述了系统的部署方式(见于图5)。

\begin{figure}[h]
  \centering
  \includegraphics[width=0.7\textwidth]{figures/physical_view.png}
  \caption{物理视图部署图}
  \label{fig:physical_view}
\end{figure}

\subsection{场景视图}
此视图通过关键的用户场景驱动和验证架构设计,确保系统满足核心需求。以下为三个核心场景:

\subsubsection{场景一:实时危险行为识别与报警}
\begin{itemize}
  \item \textbf{描述}:驾驶员行驶中长时间闭眼,客户端需在500ms内通过本地扬声器和屏幕发出声光报警。
  \item \textbf{架构验证}:
    \begin{enumerate}
      \item \textbf{物理视图}:在客户端PC上,Python客户端通过USB摄像头捕获视频,依托PC硬件执行推理,扬声器和屏幕输出报警,验证胖客户端独立性。
      \item \textbf{逻辑视图}:\texttt{DetectionClient} 模块集成 \texttt{VideoStreamer}(捕获视频)、\texttt{VideoProcessor}(预处理)、\texttt{BehaviorAnalyzer}(YOLO+MediaPipe推理)和 \texttt{AlarmManager}(触发报警),模块职责清晰。
      \item \textbf{过程视图}:对应时序图的“实时监测循环”和“触发本地声光报警”分支,验证动态流程。
      \item \textbf{开发视图}:\texttt{client/detector.py} 封装AI模型,\texttt{client/analyzer.py} 实现闭眼检测逻辑,单例模式确保模型高效复用。
    \end{enumerate}
\end{itemize}

\subsubsection{场景二:危险行为事件上报}
\begin{itemize}
  \item \textbf{描述}:场景一触发报警后,客户端将事件(时间、类型、驾驶员ID)通过HTTPS上报至后端,持久化存储,支持网络中断重试。
  \item \textbf{架构验证}:
    \begin{enumerate}
      \item \textbf{物理视图}:客户端PC与云端Linux服务器通信,Flask应用接收JSON数据,存入PostgreSQL,验证C/S架构网络部署。
      \item \textbf{逻辑视图}:\texttt{DetectionClient.DataUploader} 与 \texttt{ApiServer.EventController} 交互,\texttt{ApiServer} 通过ORM写入 \texttt{Database.EventTable},验证三层数据流。
      \item \textbf{过程视图}:对应时序图的“alt [检测到危险行为]”分支,包括重试逻辑(指数退避)。
      \item \textbf{开发视图}:\texttt{client/main.py} 调用 \texttt{requests.post},与 \texttt{server/resources/event.py} 的API端点交互,\texttt{server/models.py} 封装ORM操作。
    \end{enumerate}
\end{itemize}

\subsubsection{场景三:管理员查询历史数据}
\begin{itemize}
  \item \textbf{描述}:管理员登录Web后台,筛选并查看过去一周的“疲劳驾驶”报警记录,响应时间<5秒。
  \item \textbf{架构验证}:
    \begin{enumerate}
      \item \textbf{物理视图}:管理员通过客户端PC的Web浏览器(Chrome)访问云端Flask服务器,服务器从PostgreSQL检索数据,验证B/S架构部署。
      \item \textbf{逻辑视图}:\texttt{WebFrontend} 的 \texttt{LoginView} 和 \texttt{HistoryView} 与 \texttt{ApiServer} 的 \texttt{UserController}(认证)和 \texttt{EventController}(查询)交互。
      \item \textbf{过程视图}:如下时序图展示交互序列,验证查询流程。
      \item \textbf{开发视图}:\texttt{frontend/src/views/HistoryView.vue} 通过Axios发送GET请求至 \texttt{server/resources/event.py} 的查询端点,\texttt{server/models.py} 执行SQLAlchemy条件查询,前端组件复用性和后端查询优化降低开发成本。
    \end{enumerate}
\end{itemize}

\begin{figure}[h]
  \centering
  \includegraphics[width=0.6\textwidth]{figures/scenario_view.png}
  \caption{场景视图}
  \label{fig:scenario_view}
\end{figure}

\section{质量场景描述}
为确保架构满足关键非功能性需求,我们定义以下质量属性场景,每个场景都包含刺激、环境、响应和度量标准。

\begin{table}[H]
  \centering
  \small % 稍微缩小字体以容纳更多内容
  \caption{质量属性场景}
  \label{tab:quality_scenarios}
  \begin{tabularx}{\textwidth}{|c|X|X|X|X|X|X|X|} %<-- 核心改动
    \hline
    \textbf{场景} & \textbf{质量属性} & \textbf{来源} & \textbf{刺激} & \textbf{制品} & \textbf{环境} & \textbf{响应} & \textbf{响应度量} \\
    \hline
    1 & 性能 & 驾驶员 & 发生长时间闭眼等危险驾驶行为 & 监测客户端 & 正常运行中 & 系统发出首次警报 & 平均延迟低于 500 毫秒 \\
    \hline
    2 & 可修改性 & 开发者 & 需要增加一种新的危险行为(如“打电话检测”) & 客户端代码 & 开发阶段 & 开发者只需修改 \texttt{analyzer.py} 模块,无需改动其他模块 & 修改和测试工作量小于 4 个工时 \\
    \hline
    3 & 易用性 & 管理员 & 希望查看某驾驶员过去一周的危险行为记录 & 前端 Web 界面 & 正常登录后 & 管理员进入历史查询页面,选择条件后看到结果 & 操作步骤不超过 3 次点击,数据加载时间小于 5 秒 \\
    \hline
  \end{tabularx}
\end{table}

\section{框架与设计模式应用}
\subsection{框架与核心库选择}
\begin{itemize}
  \item \textbf{Python + OpenCV}:视频处理的事实标准,生态成熟,性能可靠。
  \item \textbf{Ultralytics YOLO}:提供高性能的实时目标检测能力,是本项目的技术基石。
  \item \textbf{MediaPipe}:用于精确的人脸关键点检测,判断眼部、口部状态,性能优于传统方法。
  \item \textbf{Flask}:轻量级、灵活的 Python Web 框架,适合快速开发 API 服务,学习曲线平缓。
  \item \textbf{Vue.js}:现代化、渐进式 JavaScript 框架,组件化开发提升前端开发效率和可维护性。
\end{itemize}

\subsection{设计模式应用}
\begin{itemize}
  \item \textbf{单例模式 (Singleton Pattern)}:在 Python 客户端中,\texttt{YOLO} 和 \texttt{MediaPipe} 模型加载耗时且占用大量内存,封装为单例,确保程序生命周期内只实例化一次。
  \item \textbf{策略模式 (Strategy Pattern)}:在 \texttt{analyzer.py} 中,将每种危险行为的判断逻辑(如 \texttt{疲劳检测策略}、\texttt{分心检测策略})封装成独立策略类,符合开闭原则。
  \item \textbf{数据访问对象/仓库模式 (DAO/Repository Pattern)}:在 Flask 后端,创建数据访问层封装数据库操作,逻辑层仅与该层交互,便于数据库切换。
  \item \textbf{观察者模式 (Observer Pattern)}:\texttt{BehaviorAnalyzer} 作为主题,检测到危险行为时通知 \texttt{AlarmManager}(本地报警)和 \texttt{DataUploader}(数据上报),实现功能解耦。
\end{itemize}

\section{优先级划分策略}
采用 \textbf{MoSCoW} 方法划分功能优先级,确保核心价值优先交付。
\begin{itemize}
  \item \textbf{Must-have (必须完成)}:构成最小可行产品(MVP)的核心功能。
    \begin{itemize}
      \item 客户端通过摄像头实时检测至少两种核心危险行为(如 \textbf{闭眼疲劳} 和 \textbf{低头分心})。
      \item 检测到危险行为后,客户端进行 \textbf{本地声光报警}。
      \item 客户端将报警事件上报给后端并存入数据库。
      \item 后端提供基础的管理员 \textbf{用户登录} 功能。
    \end{itemize}
  \item \textbf{Should-have (应该完成)}:重要功能,若时间允许优先实现。
    \begin{itemize}
      \item 前端 Web 界面以 \textbf{列表形式查询和展示} 历史报警记录。
      \item 支持更多危险行为检测(如 \textbf{手部异常}、\textbf{打电话})。
      \item 前后端用户认证机制完善。
    \end{itemize}
  \item \textbf{Could-have (可以完成)}:锦上添花的功能,优先级较低。
    \begin{itemize}
      \item 前端实现 \textbf{数据可视化图表}(如危险行为类型分布饼图)。
      \item 管理员可对报警的 \textbf{灵敏度进行配置}。
      \item 支持查询结果导出为 PDF 或 Excel。
    \end{itemize}
  \item \textbf{Won't-have (本次不做)}:明确在此版本范围之外的功能。
    \begin{itemize}
      \item 移动端 App 的开发。
      \item 远程设备管理与固件升级。
      \item 与物理硬件(GPS、4G 模块)的深度集成。
    \end{itemize}
\end{itemize}

\end{document}